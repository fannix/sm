\documentclass[12pt]{article}

\usepackage{graphics}
\usepackage{graphicx}
\usepackage{epsfig}
\usepackage{times}
\usepackage{amsmath}
\usepackage[slantfont, boldfont]{xeCJK}
\setCJKmainfont[BoldFont={STHeiti}]{STHeiti}
\setCJKsansfont[BoldFont={STHeiti}]{STHeiti}
\setCJKmonofont[BoldFont={STHeiti}]{STHeiti}


% <http://psl.cs.columbia.edu/phdczar/proposal.html>:
%
% The standard departmental thesis proposal format is the following:
%        30 pages
%        12 point type
%        1 inch margins all around = 6.5   inch column
%        (Total:  30 * 6.5   = 195 page-inches)
%
% For letter-size paper: 8.5 in x 11 in
% Latex Origin is 1''/1'', so measurements are relative to this.

\topmargin      0.0in
\headheight     0.0in
\headsep        0.0in
\oddsidemargin  0.0in
\evensidemargin 0.0in
\textheight     9.0in
\textwidth      6.5in

\title{{\bf 微博情感分析}}
\author{ {\bf 蒙新泛}  \\
北京大学\\
}

\begin{document}
\pagestyle{plain}
\pagenumbering{roman}
\maketitle

%\begin{abstract}

%The thesis proposal is a type of contract between the faculty and the student. 
%An accepted thesis proposal indicates that the work proposed by the student, 
%once completed, will be accepted by the faculty as sufficiently innovative and 
%substantial as to be recognized with the award of the degree. It is part of 
%the training of the student's research apprenticeship that the form of this 
%proposal must be as concise as those proposals required by major funding 
%agencies.

%\end{abstract}


\pagenumbering{arabic}

\section{介绍}
上一份报告中把情感倾向性(polarity)的识别当做分类问题,在这份报告中,我们把它看做一个回归问题,
即这次要识别的情感倾向性不再只有0,1,-1的取值,而是从-3到+3的取值范围。除此之外,其他的预
处理步骤,特征选取步骤仍和上次步骤保持一致。

训练数据的质量对于训练模型的影响至为关键。标注语料的人员共有五名,
我们先以图表的方式来研究他们的标注一致性。图~\ref{fig:diff}和图~\ref{fig:heatmap}
比了标注人员结果的差异。图~\ref{fig:diff}中可以明显看出标注者4与其他四名标注者
差异较大,他把绝大部分的微博都标注为正面或者负面。图~\ref{fig:heatmap}以不同颜色
代表不同标注的倾向性,由深到浅渐变颜色代表由负面到正面的倾向性。此图中将每个标注者
标注的2000条数据并排在一起比较。从中可以看出标注者2和5比较一致,其次到3和1,最差
的是4。是不是需要对标注一致性加以限制并在处理中去掉不一致的标注?这点可以考虑。

\begin{figure}[!htbp]
  \centering
  \includegraphics[width=0.9\linewidth]{../hist_diff.pdf}
  \caption{直方图(histogram)}
  \label{fig:diff}
\end{figure}

\begin{figure}[!htbp]
  \centering
  \includegraphics[width=0.8\linewidth]{../heatmap.pdf}
  \caption{热度图(heatmap)}
  \label{fig:heatmap}
\end{figure}

由于有五名标注者,所以我们将五名标注者的打分取平均作为标准答案。取平均之后,
我们再以图表的方式呈现情感倾向性的分布,如图~\ref{fig:hist}所示,
一个明显的结论是微博上的人的情感倾向略偏向于正向。

\begin{figure}
  \centering
  \includegraphics[width=0.9\linewidth]{../hist.pdf}
  \caption{情感倾向性分布}
  \label{fig:hist}
\end{figure}

\section{回归分析}
在这里,我们试验了简单的线性回归模型, Lasso回归和Ridge回归。分别用前1000(后1000条)微博作为训练数据,
来对后1000条(前1000条)微博进行预测。

经过对模型和参数的选择,能得到的最优结果用平均误差平方(mean squared eroor)
是0.34和0.31。得到的标准答案和预测值的对比图表如下面图~\ref{fig:predicted1}和~\ref{fig:predicted2}。
理想情况下,标准值
和预测值是完全一致的,在图表中应该表现为一条45度的对角线。现在我们取得的结果
不是很好,可以看出粗略线性关系,但是散度太大。
\begin{figure}
  \centering
  \includegraphics[width=0.8\linewidth]{../predict1.pdf}
  \caption{真实值-预测值}
  \label{fig:predicted1}
\end{figure}

\begin{figure}
  \centering
  \includegraphics[width=0.8\linewidth]{../predict2.pdf}
  \caption{真实值-预测值}
  \label{fig:predicted2}
\end{figure}

\section{遇到的问题和接下来的工作}
一个突出的问题是训练数据不足,我们的特征抽取过程总共识别出9246个特征,但是训练数据只有1000个,
由线性代数的知识可知,列出的方程组有9246个变量,但只有1000个方程。这样的方程求出的解
是不稳定的。
针对这个问题,我们还需要在理论上研究有相似的应用场景的问题,以及在实践中做大量的实验。
%\begin{footnotesize}
%\bibliographystyle{plain}
%\bibliography{string,itu,rfc,i-d}
%\end{footnotesize}

\end{document}



