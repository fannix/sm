\documentclass[12pt]{article}

\usepackage{graphics}
\usepackage{epsfig}
\usepackage{times}
\usepackage{amsmath}
\usepackage[slantfont, boldfont]{xeCJK}
\setCJKmainfont[BoldFont={STHeiti}]{STHeiti}
\setCJKsansfont[BoldFont={STHeiti}]{STHeiti}
\setCJKmonofont[BoldFont={STHeiti}]{STHeiti}


% <http://psl.cs.columbia.edu/phdczar/proposal.html>:
%
% The standard departmental thesis proposal format is the following:
%        30 pages
%        12 point type
%        1 inch margins all around = 6.5   inch column
%        (Total:  30 * 6.5   = 195 page-inches)
%
% For letter-size paper: 8.5 in x 11 in
% Latex Origin is 1''/1'', so measurements are relative to this.

\topmargin      0.0in
\headheight     0.0in
\headsep        0.0in
\oddsidemargin  0.0in
\evensidemargin 0.0in
\textheight     9.0in
\textwidth      6.5in

\title{{\bf 微博情感分析}}
\author{ {\bf 蒙新泛}  \\
北京大学\\
}

\begin{document}
\pagestyle{plain}
\pagenumbering{roman}
\maketitle

\begin{abstract}

The thesis proposal is a type of contract between the faculty and the student. 
An accepted thesis proposal indicates that the work proposed by the student, 
once completed, will be accepted by the faculty as sufficiently innovative and 
substantial as to be recognized with the award of the degree. It is part of 
the training of the student's research apprenticeship that the form of this 
proposal must be as concise as those proposals required by major funding 
agencies.

\end{abstract}


\pagenumbering{arabic}

\section{介绍}
上一份告中把

\begin{figure}
  \centering
  \includegraphics[width=0.9\linewidth]{../hist.pdf}
  %\caption{Different dependencies between words A and B}
\end{figure}

\begin{figure}
  \centering
  \includegraphics[width=0.9\linewidth]{../hist_diff.pdf}
  %\caption{Different dependencies between words A and B}
\end{figure}

\begin{figure}
  \centering
  \includegraphics[width=0.9\linewidth]{../heatmap.pdf}
  %\caption{Different dependencies between words A and B}
\end{figure}

\begin{figure}
  \centering
  \includegraphics[width=0.9\linewidth]{../predict1.pdf}
  %\caption{Different dependencies between words A and B}
\end{figure}

\begin{figure}
  \centering
  \includegraphics[width=0.9\linewidth]{../predict2.pdf}
  %\caption{Different dependencies between words A and B}
\end{figure}
%\begin{footnotesize}
%\bibliographystyle{plain}
%\bibliography{string,itu,rfc,i-d}
%\end{footnotesize}

\end{document}



